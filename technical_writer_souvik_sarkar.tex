%%%%%%%%%%%%%%%%%%%%%%%%%%%%%%%%%%%%%%%%%
% Medium Length Professional CV
% LaTeX Template
% Version 2.0 (8/5/13)
%
% This template has been downloaded from:
% http://www.LaTeXTemplates.com
%
% Original author:
% Trey Hunner (http://www.treyhunner.com/)
%
% Revision author (current version):
% Souvik Sarkar (GitHub: @sounix000)
%
% Important note:
% This template requires the resume.cls file to be in the same directory as the
% .tex file. The resume.cls file provides the resume style used for structuring the
% document.
%
%%%%%%%%%%%%%%%%%%%%%%%%%%%%%%%%%%%%%%%%%

%----------------------------------------------------------------------------------------
%	PACKAGES AND OTHER DOCUMENT CONFIGURATIONS
%----------------------------------------------------------------------------------------

\documentclass{resume} % Use the custom resume.cls style

\usepackage{amsfonts}
\usepackage{bookmark}
\usepackage[left=0.75in,top=0.6in,right=0.75in,bottom=0.6in]{geometry} % Document margins
\usepackage{hyperref}
\hypersetup{
    colorlinks=true,
    linkcolor=black,
    urlcolor=blue
}

\name{SOUVIK SARKAR} % Your name
\address{\bf Senior Technical Writer } % Your profession
\address{\bf \href{mailto:sounix000@gmail.com}{sounix000@gmail.com} \\ (+91) 6360229130 \\ \href{https://sounix000.githu.io}{sounix000.github.io} } % Your email, phone number, and website

\begin{document}

%----------------------------------------------------------------------------------------
%	PROFESSIONAL SUMMARY SECTION
%----------------------------------------------------------------------------------------

\begin{rSection}{Overview}
Technical Writer with 8 years of experience documenting software and APIs for developers and administrators, backed by 4 years of experience as a developer. I develop technical content for an optimal developer experience, using docs-as-code, modular documentation, and open-source tools.
\end{rSection}

%----------------------------------------------------------------------------------------
%	SKILLS SECTION
%----------------------------------------------------------------------------------------

\begin{rSection}{Skills}

\begin{tabular}{ @{} >{\bfseries}l @{\hspace{6ex}} l }
Programming & C/C++, Python, Bash, JavaScript  \\
Markup & AsciiDoc, Markdown, XML, HTML \\
Deployment & GitLab, Docker, Kubernetes \\
Cloud & Google Cloud Platform \\
Images/Diagrams & Inkscape, Mermaid.js \\  
Code/Text Editors & VS Code, Vim \\
Projects and Analytics & JIRA, Google Sheets \\

\end{tabular}

\end{rSection}


%----------------------------------------------------------------------------------------
%	WORK EXPERIENCE SECTION
%----------------------------------------------------------------------------------------

\begin{rSection}{Employment}

\begin{rSubsection}{SUSE}{March 2023 - Present}{Senior Technical Writer}{}
\item Own the SUSE Linux Enterprise guides for performance tuning, virtualization, and compliance certifications.
\item Contribute to transitioning from monolithic guides to topic-based modular documentation for SUSE Linux Enterprise.
\item Collaborate with engineers, release managers, and product managers to realign the SUSE Linux Enterprise documentation with the upstream OpenSUSE documentation.
\item Interface with the support engineering team to reinforce existing documentation and create client-specific knowledge base articles.
\item Participate in peer reviews and mentor new technical writers in the team. 
\item \underline{Technologies/Tools}: SUSE Linux Enterprise, OpenSUSE Leap, Bash, Python, DocBook XML, AsciiDoc, GitHub, VS Code, SUSE style guide, JIRA, Confluence Wiki, Slack. 
\end{rSubsection}

\begin{rSubsection}{Red Hat}{November 2020–March 2023}{Senior Technical Writer}{}
\item Lead documentation efforts for OpenShift CI/CD Pipelines, including triaging issues, creating documentation plans, and deciding content strategy.
\item Participated in peer reviews of documents for other products and mentored new technical writers in the team. 
\item Collaborated with content strategists and product managers to realign product documentation with upstream documentation.
\item Interfaced with the support engineering team to reinforce existing documentation and create client-specific knowledge-base articles.
\item Data analyzed to improve documentation coverage, flow, architecture, findability, and accessibility. 
\item \underline{Technologies/Tools}: Fedora Linux, Docker, Kubernetes, Golang, Node.js, Bash, AsciiDoc, Markdown, Lucidchart, GitHub, VS Code, Atom, IBM style guide, JIRA, Trello, Tableau, Google Sheets, Slack. 
\end{rSubsection}

\begin{rSubsection}{Ribbon Communications}{January 2019–November 2020}{Senior Technical Writer}{}
\item Lead documentation efforts for SBC Core, resulting in improved customer satisfaction ratings.
\item Developed documents such as CLI reference and REST API user guide.
\item Participated in peer reviews of documents for other products and mentored new technical writers in the team.
\item Collaborated to revamp the internal style guide for easier onboarding of new technical writers.
\item \underline{Technologies/Tools}: Proprietary networking and telecom hardware, Debian Linux, C++, Bash, OpenStack, AWS, Confluence Wiki, Snagit, Microsoft style guide, JIRA.
\end{rSubsection}

%----------------------------------------------------------------------------------------
%	BLANK SPACE
%\vspace{1cm}
%\hspace{1cm}
%----------------------------------------------------------------------------------------

%------------------------------------------------

\begin{rSubsection}{FusionCharts (Infosoft Global Pvt. Ltd.)}{March 2018–January 2019}{Technical Content Writer}{}
\item Developed and reviewed sample applications, code snippets, and content for developer documentation.
\item Implemented automation scripts to improve the operational efficiency of pre-release workflows.
\item Worked with the engineering and marketing teams to create and publish technical articles on third-party websites and blogs.
\item \underline{Technologies/Tools}: macOS, JavaScript (Node.js, Vue.js, React.js), Markdown, Lucidchart, Google Style Guide, GitHub, Trello, Slack.
\end{rSubsection}

%------------------------------------------------

\begin{rSubsection}{Tata Elxsi}{June 2016–March 2018}{Consultant Technical Writer}{Contractor for Ribbon Communications}
\item Created and maintained Quick Start tutorials, API guides, User guides, and Release Notes.
\item \underline{Technologies/Tools}: Proprietary networking and telecom hardware, Debian Linux, Confluence Wiki, Snagit, Microsoft style guide, JIRA.
\end{rSubsection}

%------------------------------------------------

\begin{rSubsection}{Agaze Technologies}{September 2015–May 2016}{Software Engineer}{}
\item Developed internal and client-facing applications, focusing on back-end implementation.
\item \underline{Technologies/Tools}: Ubuntu Linux, Python (Django, Flask), PostgreSQL, JavaScript (Node.js, React.js), Google Cloud Platform (App Engine).
\end{rSubsection}

%------------------------------------------------

\begin{rSubsection}{Freelance/Self-employed}{July 2012–August 2015}{Web Developer}{}
\item Developed websites, administered systems, and wrote technical content for blogs as a ghostwriter.
\item \underline{Technologies/Tools}: Ubuntu Linux, Python (Flask), MySQL, JavaScript.
\end{rSubsection}

\end{rSection}

%----------------------------------------------------------------------------------------
%	AWARDS SECTION
%----------------------------------------------------------------------------------------

\begin{rSection}{Recognitions}

\item {\bf Red Hat} - Several peer-to-peer awards from team members (including engineering) for collaboration and commitment to excellence.
\item {\bf Ribbon Communications} - Motivate Award (2019-2020) for extraordinary performance.
\item {\bf Tata Elxsi} - Customer Appreciation Award (2016-2017) based on client recommendation.

\end{rSection}


%----------------------------------------------------------------------------------------
%	PORTFOLIO SECTION
%----------------------------------------------------------------------------------------

%\begin{rSection}{Portfolio}

%\item {\bf Blog -} sounix000.github.io/souviksarkar/

%\item {\bf GitHub -} github.com/sounix000/

%\end{rSection}


%----------------------------------------------------------------------------------------
%	EDUCATION SECTION
%----------------------------------------------------------------------------------------

\begin{rSection}{Education}

\begin{rSubsection}{Bachelor of Technology}{2014}{Applied Electronics and Instrumentation Engineering}{West Bengal University of Technology}
\end{rSubsection}

\end{rSection}

\end{document}
